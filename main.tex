\documentclass[UTF8]{article}
\usepackage{amsmath,amssymb,amsthm}
\usepackage{mathrsfs}
\usepackage{ctex}
\usepackage{graphicx}
\usepackage{float}


\newcommand{\D}{\mathrm{d}}
\newcommand{\E}{\mathrm{e}}
\newcommand{\im}{\mathrm{i}}


\begin{document}
	
\section*{0 小振动}

在这一章里如果没有求和号就是使用爱因斯坦求和约定.

\subsection*{0 多自由度力学体系的谐振}

	考虑一个有$n$个自由度的保守力学系统,其拉格朗日量为$L(q_i,\dot{q}_i)=T-V$,势能$V(q_i)$.其中$q_i,\dot{q_i}$是广义坐标和广义速度.
	
	这里我们感兴趣的是在稳定的平衡位置附近的运动.设系统的一个平衡位置为$q_{i0}$,则必有
	
	\[\left.\frac{\partial V}{\partial q_i}\right|_{q_i=q_{i0}}=0\]
	
	不妨把广义坐标齐次化,令$x_i:=q_i-q_{i0}$,则:
	
	\[\left.\frac{\partial V}{\partial x_i}\right|_{x_i=0}=0\]
	
	那么在平衡位置附近可以将势能进行Taylor展开
	
	\[V(x)=V_0+\frac{1}{2}k_{ij}x_ix_j+o(|x|^2)\]
	
	设$x=x_ie_i$,那么也可以写成矩阵的形式
	
	\[V(x)=V_0+\frac12x^Tkx\]
	
	而动能$T$本身就是$\dot{x}$的二次型$T=\frac12\dot{x}^Tm\dot{x}$,我们关心的运动的性质也足够良好以至于$\dot{x}$也是小量,故可取$m$为平衡位置的取值.
	
	那么系统的拉格朗日函数即
	
	\[L=\frac12(\dot{x}^Tm\dot{x}-x^Tkx)\]
	
	对应的动力学方程
	
	\[m_{ij}\ddot{x_i}+k_{ij}x_i=0\]
	
	现在我们来讨论谐振动解的存在性与完备性.直接假设这个系统具有简谐振动形式的解$x=c\E^{\im \omega t}$.那么$\ddot{x}=-\omega^2 x$,动力学方程化为
	
	\[(k_{ij}-\omega^2m_{ij})c_i=0\]
	
	也即
	
	\[(k-\omega^2m)c=0\]
	
	这样我们将微分方程的求解转化为代数方程的求解,即上述$n$元方程组的求解.
	
	方程组有非零解要求系数行列式为0,即
	
	\[\det(k-\omega^2m)=0\]
	
	二次型的对称性要求$m,k$都是实对称矩阵,故一定存在正交矩阵$P$,使得
	
	\[P^TmP=\mathrm{diag}\{m_1,m_2,\cdots,m_n\}=M\quad\Leftrightarrow\quad m=PMP^T\]
	
	再记$D=\mathrm{diag}\{\sqrt{m_1},\sqrt{m_2},\cdots,\sqrt{m_n}\}$(假设$m_i\ne0$,$D$可逆,这要求动能在平衡点非退化).那么系数矩阵满足关系
	
	\[k-\omega^2m=PD(D^{-1T}P^TkPD^{-1}-\omega^2I)D^TP^T\]
	
	故必有
	
	\[\det(k-\omega^2m)=m_1m_2\cdots m_n\det(D^{-1T}P^TkPD^{-1}-\omega^2I)\]

	注意到$D^{-1T}P^TkPD^{-1}$也是实对称矩阵,故其可以正交对角化(其实到这一步已经结束了),即存在正交矩阵$Q$,使得
	
	\[Q^T(D^{-1T}P^TkPD^{-1})Q=\mathrm{diag}\{\omega_1^2,\omega_2^2,\cdots\omega_n^2\}=\Omega\]
	
	平衡位置稳定要求$k$为正定矩阵,而合同变换不改变矩阵的正定性,所以$\omega_i^2$的记法是合理的.
	
	故
	
	\[k-\omega^2m=PDQ(\Omega-\omega^2I)Q^TD^TP^T\]
	
	\[\det(k-\omega^2m)=m_1m_2\cdots m_n\det(\Omega-\omega^2I)\]
	
	(说实话写到这里我突然意识到这不就是线代上的正定矩阵与实对称矩阵的同时对角化问题吗......失败大学生石锤了orz)
	
	上述推导表明$(k-\omega^2m)c=0$等价于$(\Omega-\omega^2I)Q^TD^TP^Tc=0$.再进行非退化代换$C=Q^TD^TP^Tc$,则等价于$(\Omega-\omega^2I)C=0$.
	
	而每一个特征向量对应$\cos\omega t$和$\sin \omega t$二线性无关解.记这些特征向量为系统的简正坐标$Q_i=C_{ij}x_j$.
	
	可见,在非退化的情况下,系统的确存在$2n$个线性无关的振动解,并且具有不超过$n$个共振频率.考虑到该线性常微分方程组解空间的维数只有$2n$,我们这里找到了系统的通解.
	
	至于具体的振动模式/特征向量(\textbf{简正模})的求解,从上面的推导可以看出直接按照线性代数中求解矩阵特征值与特征向量的解法即可.顺便提一下,由于奇奇怪怪的历史原因,刚刚出现的特征方程$|\omega^2m-k|=0$和物理上出现的其他类似的特征方程都叫\textbf{久期方程},至于为什么,向最近的记录神甫提问吧(逃)
	
	\newpage
	
	
	
	
	
	
	
	\subsection*{1 摄动法求解非谐振子}
	
	虽然上面得到了很优美的结论,但是我们不应忘记这是在Taylor展开取一阶项的条件下得到的.从一般的振动的意义下,这可以说是trivial而不符合万机之神精妙的创造(雾)的.对于振幅较大而非线性效应不可忽略的情况,我们也应予讨论.
	
	但在这之前,我们先来点数学(并不)上的铺垫:微扰法.
	
	举一个简单的例子(这是归纳法!科学方法!):考虑如下方程的求根
	
	\[x-1-\varepsilon x^3,\quad\varepsilon\ll1\]
	
	我们考虑求其在$x=1$附近的根,显然这是$\varepsilon$的函数.根据机械教圣典物理篇万物皆可Taylor的教义,我们将其展开为Taylor级数
	
	\[x=x^{(0)}+x^{(1)}\varepsilon+x^{(2)}\varepsilon^2+\cdots\]

	然后我们把这个级数代回到原方程,根据Taylor展开的唯一性比较$\varepsilon$对应项的系数,我们就可以得到一系列的方程,进而以任意高的精度解出$x^{(n)}$.
	
	具体操作上,显然零阶下有$x^{(0)}=1$.那么接下来我们保留到一阶项,代入得
	
	\[x^{(1)}\varepsilon-\varepsilon=0 \Rightarrow x^{(1)}=1\]
	
	计算出一阶项后,再保留到二阶项,得
	
	\[x^{(2)}\varepsilon^2-3\varepsilon^2=0\Rightarrow x^{(2)}=3\]
	
	以此类推,我们可以得出

	\begin{figure}[H]
		\centering
		\includegraphics[width=1.25\linewidth]{pics/QQ截图20210720221755}
		\label{fig:qq20210720221755}
	\end{figure}
	
	(x)
	
	方程的另外两个跟当$\varepsilon\to 0$时趋于无穷,因而正则摄动(就是刚刚的操作)不适用,我们可以使用奇异摄动法来进行求解.具体来说,就是进行变量替换$x=\frac{\xi}{\varepsilon^\mu}$,代入方程可得
	
	\[\xi\varepsilon^{2\mu-1}-\varepsilon^{3\mu-1}-\xi^3=0\]
	
	考虑到$\varepsilon^{3\mu-1}$必然比其他两项小到不知哪里去了,舍去得
	
	\[\xi\varepsilon^{2\mu-1}-\xi^3=0\]
	
	回想起来我们进行这个变量代换的目的就是消掉$x$的发散,那么对机神的信仰告诉我们(?)$\xi(\varepsilon)$的零阶项必不为0.从而系数可能相等要求有$2\mu-1=0$,即$\mu=\frac{1}{2}$
	
	现在回到代换后的原始方程
	
	\[\xi-\varepsilon^{1/2}-\xi^3=0\]
	
	这个方程我们可以按照正则摄动进行展开求解(注意要按$\varepsilon^{1/2}$的幂次).一通操作后我们得到$x$的展开式(之一)
	
	\begin{figure}[H]
		\centering
		\includegraphics[width=1\linewidth]{pics/QQ截图20210720223418}
		\label{fig:qq20210720223418}
	\end{figure}
	
	至于什么收敛不收敛合法不合法的,那肯定是受了混沌的蛊惑对机神信仰不足才会发生的事,高阶的物理神甫甚至会使用渐近级数这种收敛到数学地狱里面的阴间操作,那么只好额叶切除了,请(悲)
	
	需要指出的是fx老师强调现代的计算机数值方法非常发达,而摄动法的收敛性与收敛速度也是有限的,如果你发现微扰摄动要算到几十几百项才能得到一个比较好的解........那你还是赶紧安抚一下计算机的机魂让他赶紧帮你算点数值解或者什么格点方法吧(悲)
	
	现在我们来考虑具体的动力学系统.不妨先考虑把一般情况下的拉格朗日函数展开到第一阶非线性项,也即三阶项.选取之前得到的简正坐标为广义坐标,那么拉格朗日函数具有如下形式:
	
	\[L=\frac12 (\dot{Q_i}^2-\omega_i^2Q_i^2)+\frac12 \lambda_{ijk}Q_i\dot{Q_j}\dot{Q_k}+\frac12 \mu_{ijk}Q_iQ_jQ_k\]
	
	并且对应的动力学方程满足如下形式
	
	\[\ddot{Q_i}+\omega_i^2Q_i=f(Q,\dot{Q},\ddot{Q})\]
	
	其中$f$为二次齐次函数.
	
	既然我们只将拉格朗日函数进行了额外的一次展开,我们就来求一阶修正:
	
	\[Q_i=Q_i^{(0)}+Q_i^{(1)}\]
	
	对这个神圣定义的解读是其中第一项为将$f$视为0的解,第二项为非线性因素引起的修正,且$Q_i^{(1)}\ll Q_i^{(0)}$.
	
	那么我们发现一阶项满足的方程
	
	\begin{align*}
		\ddot{Q_i}^{(1)}+\omega_i^2Q_i^{(1)}&=a_ia_j\cos(\omega_i t+\phi_i)\cos(\omega_j t+\phi)\\
		&=\frac12a_ia_j\{\cos[(\omega_i+\omega_j)t+(\phi_i+\phi_j)]+\cos[(\omega_i-\omega_j)t+(\phi_i-\phi_j)]\}
	\end{align*}
	
	可以看出非齐次项含有$\omega_i\pm\omega_j$项,也即一阶修正项中产生了系统固有频率和差的次级频率.这些频率称为\textbf{组合频率}.
	
	进一步我们可以考虑更高阶的修正.但是如果我们照搬上面的方法,不难发现二阶近似解中就会出现$\omega_i+\omega_j-\omega_i=\omega_i$项,解的振幅随时间增长,显然是不物理的.
	
	发生这种亵渎知识的事情,是因为刚刚的做法对欧姆尼赛亚的信仰不足(划掉)没有考虑高阶非线性项对本征频率的修正.
	
	这可以在一维系统的情况下考虑,力学中可以证明此时有界运动必然是周期性的,从而可以展开成基频对应的Fourier级数.然而如果忽略本征频率的改变,那么Fourier展开的条件被破坏,无法保证级数收敛到真实运动(事实上应该是一定不收敛?).所以在考虑高阶修正时必须从一开始就考虑本征频率的改变.
	
	不妨以一个一维的非谐振子为例.一般地,我们考虑一个光滑的势函数,将其在平衡位置对$x$展开成Taylor级数,则拉格朗日函数
	
	\[L=\frac{1}{2}m\dot{x}^2-\frac{1}{2}m\omega_0^2x^2-\sum_{n=3}^{\infty}\frac{1}{n+1}f_nx^{n+1}\]
	
	对应的动力学方程
	
	\[\ddot{x}+\omega_0^2x=-\sum_{n=2}^{\infty}f_nx^n\]
	
	(不会吧不会吧不会有人在这个跟field theory没有半毛钱关系的地方还要用Einstein求和吧)
	
	为了使用摄动法,我们考虑要以之进行展开的量.既然势能是对位移$x$进行展开的,我们应选取$x$的一个特征尺度为小量,如某种意义下的振幅.这样,用无量纲的$\varepsilon$标示小量的阶数,我们可以设初始条件
	
	\[x|_{t=0}=\varepsilon A,\quad\dot{x}|_{t=0}=0\]
	
	将$x$有展开成类似幂级数形式的解:
	
	\[x=\varepsilon x^{(1)}+\varepsilon^2 x^{(2)}+\varepsilon^3 x^{(3)}+\cdots\]
	
	显然$x^{(0)}=0,x^{(1)}=A\cos \omega t$.
	
	现在再将$\omega$展开:
	
	\[\omega=\omega^{(0)}+\varepsilon\omega^{(1)}+\varepsilon^2\omega^{(2)}+\cdots\]
	
	显然$\omega^{(0)}=\omega_0$.
	
	这时我们也将初始条件展开,可以得到
	
	\[x^{(1)}|_{t=0}=A,x^{(k)}|_{t=0}=0(k>1);\quad \dot{x}^{(k)}|_{t=0}=0(k>0)\]
	
	先保留到二阶小量(即$\varepsilon$的二次项),代入动力学方程,化简得:
	
	\[
		\begin{aligned}
			\ddot{x}^{(1)}+\omega_0^2 x^{(1)}
			&=-f_2A^2\cos^2\omega t+2\omega_0\omega^{(1)}A\cos\omega t\\
			&=-\frac{1}{2}f_2A^2-\frac{1}{2}f_2A^2\cos2\omega t+2\omega_0\omega^{(1)}A\cos\omega t
		\end{aligned}
	\]
	
	等式中不出现共振项(好像也叫久期项?不管了翻译出来挨打)要求$\omega^{(1)}=0$,与之前的讨论相符.于是现在二阶方程变为	
	
	\[\ddot{x}^{(2)}+\omega_0^2 x^{(2)}=-\frac{1}{2}f_2A^2-\frac{1}{2}f_2A^2\cos2\omega t\]
	
	相信不是物理学院的同学肯定都会解这个非齐次线性微分方程(物理学院的同学还没学orz!),随便用点Laplace变换或者什么方法很容易解出
	
	\[x^{(2)}=(-\frac{1}{2}+\frac{1}{3}\cos\omega+\frac{1}{6}\cos2\omega t)\frac{f_2A^2}{\omega_0^2}\]
	
	(这里Landau直接忽略了初始条件orz但是如果不忽视初始条件好像严格说要跑出个$\omega_0$?我也不清楚orz不过偷偷换成$\omega$了,至少在三阶近似下没影响到频移)
	
	接下来保留到三阶小量,注意到$x^{(1)}$的频移对三阶项无贡献,化简得
	
	\[
	\begin{aligned}
		\ddot{x}^{(3)}+\omega_0^2x^{(3)}=&-\frac{f_2^2A^3}{6\omega_0^2}(2-5\cos\omega t+2\cos2\omega t+\cos3\omega t)\\
		&-\frac{f_3A^3}{4}(3\cos\omega t+\cos3\omega t)+2A\omega_0\omega^{(2)}\cos\omega t
	\end{aligned}	
	\]
	
	令共振项系数为0,得
	
	\[\omega^{(2)}=\frac{5f_2^2A^2}{12\omega_0^3}-\frac{3f_3A^2}{8\omega_0}\]
	
	可见,高阶非线性效应使得系统的频率发生改变.并且频率与振幅有关,这是与线性谐振子非常不同的.
	
	这个计算我们还可以继续到任意阶的精度,考虑到继续计算意义不大,我们在此打住.
	
	由微扰摄动的计算过程可以看出,非线性效应对小振动的力学系统能够带来的一般影响有:
	
	\begin{enumerate}
		\item 系统固有频率改变,并且随振幅变化
		\item 系统的运动偏离简单的正弦函数,倍频的傅里叶系数不为0
	\end{enumerate}
	
	另外一个很好玩的现象是非线性谐振子的受迫振动,正好是cupt2021的一道题,具体而言就是一定驱动力大小时特定的驱动力频率下系统可能的受迫振动解的振幅不唯一,而分裂为3个分立的解,体现了欧姆尼赛亚创造的精妙,可以说是十分神奇的非线性现象了.具体可以参阅参考书\cite{Landau}29节.
	
	最后,我知道lc教授是一个非常敬业且水平极高的好老师,希望他能原谅我在这里yygq他的书(逃).
	
	\newpage





	
	
	\subsection*{10 参数共振}
	
	如果你对上面这个标题编号感觉奇怪,说明你还没有掌握神圣的二进制语,自裁,请(x)
	
	现在我们考虑另一种奇怪的谐振子,它的动力学方程可以写成如下形式:
	
	\[\ddot{x}+\omega^2(t)x=0)\]
	
	(不会吧不会吧不会有人在classical mechanics的范围还真的想Lagrangian从头做到尾吧?)
	
	如果$\omega(t)$是周期函数,那么这样的振子被称为参数振子(paramatric oscillator).设周期为$T$.
	
	注意到这是一个二阶常微分方程,那么它解空间的维数为2,也即存在两个线性无关的基本解$x_1,x_2$,并且二者的线性组合生成了方程的全部解.
	
	又由方程的平移对称性,$x_1(t+T),x_2(t+T)$也是方程的解,所以二者都可以由基本解线性表出.具体的表出系数取决于基的选取.通常的选取方法是所谓的标准基(standard basis):
	
	\[
	\begin{matrix}
		x_1(0)=1,&\dot{x_1}(0)=0\\
		x_2(0)=0,&\dot{x_1}(0)=1
	\end{matrix}
	\]
		
	在这种情况下,不难验证$(x_1(t+T),x_2(t+T))^T$由$(x_1(t),x_2(t))^T$表出的矩阵$R$(这称为Floquet矩阵/算符)具有简单的形式:
	
	\[
		\begin{pmatrix}
			x_1(t+T)
			\\x_2(t+T)
		\end{pmatrix}
		=
		\begin{pmatrix}
			x_1(T)&\dot{x_1}(T)\\
			x_2(T)&\dot{x_2}(T)
		\end{pmatrix}
		\cdot
		\begin{pmatrix}
			x_1(t)
			\\x_2(t)
		\end{pmatrix}
	\]
	
	即
	
	\[
		R=
		\begin{pmatrix}
			x_1(T)&\dot{x_1}(T)\\
			x_2(T)&\dot{x_2}(T)
		\end{pmatrix}
	\]
	
	这样,我们只要解出微分方程在$[0,t]$的解,就可以通过Floquet算符的重复作用以得到任意时间的$x$.
	
	为了研究Floquet算符幂次的性质,我们应求出其特征值.我们有本征方程
	
	\[\lambda^2-(x_1(T)+\dot{x}_2(T))\lambda+\det(R)=0\]
	
	注意到$\det(R)$就是解的Wronski行列式.根据常微分方程的Liouville定理,$W(T)=W(0)\exp(-\int_{0}^{T}p(x)\D x)=W(0)=1$,故本征方程

	\[\lambda^2-(x_1(T)+\dot{x}_2(T))\lambda+1=0\]
	
	这意味着二本征值的乘积为1.而$x_1(T)+\dot{x}_2(T)$则取决于$\omega(t)$的具体表现.从而我们可以讨论特征值的性质,进而讨论长时间下解的行为.至于怎么从$\omega(t)$确定$x_1(T)+\dot{x}_2(T)$,日后再说,问就是信仰不足(x)
	
	情况1:二本征值为共轭的单位圆上的复数.
	
	这对应着判别式小于0的情况,即$|x_1(T)+\dot{x}_2(T)|<2$.
	
	动动脑子(或者伺服颅骨)就知道这个时候两个解在$(-\infty,\infty)$上都是有界的,并且搞不好本征值对应的旋转的频率与$\omega$的频率满足一定比例那还可以直接出现周期解,trivial,trivial
	
	情况2:二本征值是不等的实数.
	
	这对应判别式大于0的情况.
	
	此时解空间中存在两个基函数,其中一个振幅按周期$T$指数增长,另一个指数衰减.
	
	情况3:二本征值简并,对应$x_1(T)+\dot{x}_2(T)=\pm2$.此时本征值必然为$\pm 1$.
	
	这种情况下$R$又分为可对角化与不可对角化两种情况.可对角化的作为第一种情况的退化,显然是trivial又trivial的,我们主要考虑不可对角化的情况.
	
	这时候Floquet算符在某组基下的矩阵为其Jordan标准型
	
	\[R=\begin{pmatrix}1&1\\0&1\end{pmatrix}\]

	作用n次后,有
	
	\[R^n=\begin{pmatrix}1&n\\0&1\end{pmatrix}\]

	也出现了无界的情况.也即解的振幅随时间线性发散,是第二种情况的退化.
	
	利用以上的讨论可以指导我们荡秋千,详见参考书\cite{LiuChuan}相应节的讨论.
	
	\newpage
	
	
	
	
	
	
	
\section*{1 刚体运动}
	
	这一节的讨论是经典到不能再经典的.具体的原因可以参见\cite{LiuChuan}P112的脚注.
	
\subsection*{0 转动的数学表述}
	
	刚体嘛,就是无穷多个满足$|x_i-x_j|=const,\forall i,j$的质点组.当然这不是很严谨的定义,但是大家都能理解,就意思一下,意会即可.
	
	绝大多数刚体有6个自由度.当然,也不是没有自由度少的,比如一维刚性杆的自由度就只有5,绕长轴旋转的自由度没了(当然你也可以说它还在,但是它也不参与任何相互作用,所以不如说它不存在)
	
	描述刚体运动时,我们常常会选择原点位于质心,一个与刚体一起运动,固连在刚体上的坐标架.这称为刚体的\textbf{体坐标架}(body axis).然而这并没什么卵用,加速度惯性力什么的麻烦的要死,所以我们一般只是意思意思,把一般参考系中的矢量投影到刚体坐标系的基上来方便表述而已.
	
	讨论转动时,可以采用两种等价的描述:一种视为矢量本身不变,而转动空间的基矢,相当于在空间中进行基变换(所谓的旋转矩阵就是过渡矩阵的转置),这称为被动观点;另一种是直接转动矢量,这时旋转矩阵就直接是线性变换的矩阵.显然对于同一个旋转变换,这两种观点的旋转角是相反的.
	
	现在我们来讨论旋转变换的表述.首先,出于空间均匀性的考虑,它应该得是个线性变换.然后,它要使刚性约束的形式不变,也即保欧氏内积,故它得是正交变换.
	
	两个正交变换复合还是正交变换,故所有正交变换在线性变换的复合下构成一个群.也即全体正交矩阵在矩阵的乘法下构成一个群,称为三维正交群$\mathrm{O(3)}$.不难证明正交矩阵的行列式必为$\pm1$,那么我们可以将正交矩阵再依行列式的正负分为两类.称行列式为正的为\textbf{正常转动},反之为非正常转动.可以证明,全体非正常转动都可以表示为一个正常转动与$-I$的积.不难看出全体正常转动构成了$\mathrm{O(3)}$的一个子群,称为特殊三维正交群$\mathrm{SO(3)}$,而非正常转动不能.
	
	关于三维正常转动,存在一个重要定理:Euler定理.
	
	\newtheorem*{Euler}{Euler定理}
	
	\begin{Euler}
		对于任意一个三维正常转动,空间中必然存在一个特殊方向,该方向上的矢量在转动作用下不变.
	\end{Euler}
	
	事实上,这可以推广到任意奇数维的空间中.可以从正规变换在$\mathbb{R}$的一般形式加上保欧氏内积的限制条件直接证明.
	
	于是我们可以用个3自由度完全描述三维转动:一个是转动的角度,两个是旋转轴的方向.我们记以单位向量$\boldsymbol{n}$为转轴,旋转了$\Theta$角的转动为$R(\Theta,\boldsymbol{n})$.那么我们可以讨论$\mathrm{SO(3)}$的拓扑结构.考虑$\mathbb{R}^3$中的一个球心位于原点的半径为$\pi$的球.对于$\Theta>0$的情况,约定球中的点表示以位矢对应的方向为$\boldsymbol{n}$,到原点距离为$\Theta$的转动.而$\Theta<0$时则等同于为$R(-\Theta,-\boldsymbol{n})$.至于球面上的对称点,考虑到$R(\pi,\boldsymbol{n})=R(\pi,-\boldsymbol{n})$,应视为等同.这个复杂的构造就是$\mathrm{SO(3)}$群流形的拓扑结构.在数学上称为三维射影空间(three-dimensional projective space),记为$\mathrm{RP^3}$.(完全看不懂,大学生就我最失败orz)
	
	现在讨论无穷小转动.个人认为无穷小转动可以认为是与单位变换之差的范数无穷小(分析的意义上)的转动.具体展开来,对于一个无穷小转动,我们可以将其矩阵表示为:
	
	\[A=I+\D\theta_i(\im S_i)\]
	
	其中$\D\theta_i$分别对应$xy,yz,zx$平面的无穷小角位移,$S_i$是如下三个反对称矩阵:
	
	\[
	S_1=
	\begin{pmatrix}
		0&0&0\\
		0&0&\im\\
		0&-\im&0
	\end{pmatrix}
	,\quad S_2=
	\begin{pmatrix}
		0&0&-\im\\
		0&0&0\\
		\im&0&0
	\end{pmatrix}
	,\quad S_3=
	\begin{pmatrix}
		0&\im&0\\
		-\im&0&0\\
		0&0&0
	\end{pmatrix}
	\]
	
	$S_i$一般被称为三维转动的\textbf{生成元}(generator),它们能够"积分"生成$\mathrm{SO(3)}$这个Lie群中的元素.
	
	可以发现这三个生成元之间的对易关系与量子力学的对易关系相同.神奇.
	
	除了刚刚那堆奇奇怪怪的东西以外,还可以用\textbf{Euler角}来描述转动.
	
	不妨设固定坐标架$XYZ$,那么三个Euler角$\phi,\theta,\psi$描述的转动是这样三个顺序转动的叠加:
	
	1.绕原本的$Z$轴逆时针旋转$\phi$
	
	2.绕新的$X_1$轴逆时针旋转$\theta$
	
	3.绕新的$Z_2$轴逆时针旋转$\psi$
	
	最终得到体坐标架$xyz$.
	
	不难看出三次转动的矩阵分别是
	
	\[
	\begin{pmatrix}
		\cos\phi&\sin\phi&0\\
		-\sin\phi&\cos\phi&0\\
		0&0&1
	\end{pmatrix}
	,
	\begin{pmatrix}
		1&0&0\\
		0&\cos\theta&\sin\theta\\
		0&-\sin\theta&\cos\theta
	\end{pmatrix}
	,
	\begin{pmatrix}
		\cos\psi&\sin\psi&0\\
		-\sin\psi&\cos\psi&0\\
		0&0&1
	\end{pmatrix}
	\]
	
	将三个变换相乘,得到与欧拉角$(\psi,\theta,\phi)$对应的旋转的矩阵为
	
	\[
	\begin{pmatrix}
		\cos \psi \cos \phi-\cos \theta \sin \phi \sin \psi & \cos \psi \sin \phi+\cos \theta \cos \phi \sin \psi & \sin \psi \sin \theta \\
		-\sin \psi \cos \phi-\cos \theta \sin \phi \cos \psi & -\sin \psi \sin \phi+\cos \theta \cos \phi \cos \psi & \cos \psi \sin \theta \\
		\sin \theta \sin \phi & -\sin \theta \cos \phi & \cos \theta
	\end{pmatrix}
	\]
	
	"读者可以验证这个矩阵满足$A^T=A^{-1}$"emmmmm.............我承认我的生物湿件没法达到神圣机械的高度
	
	一个重要的结果就是用Euler角及其对时间的微商来描述刚体的角速度.由几何关系易得动坐标系下
	
	\[
	\begin{matrix}
		\Omega_x=\dot{\phi}\sin\theta\sin\phi+\dot{\theta}\cos\phi\\
		\Omega_y=\dot{\phi}\sin\theta\cos\phi-\dot{\theta}\sin\phi\\
		\Omega_z=\dot{\phi}\cos\theta+\dot{\psi}
	\end{matrix}
	\]
	
	三维空间的转动还有一种表示方法是Caylay-Klein参数.其核心是将$\mathbb{R}^3$与二维无迹Hermite矩阵同构,此时矩阵的行列式对应$\mathbb{R}^3$中向量的模方.而幺正(酉)相似不改变矩阵的行列式(一般的相似即可,这是单纯避免一对多无法形成映射?),所以幺正相似能够诱导一个三维的正常转动.这建立了一个$\mathrm{SO(3)}$到$\mathrm{SU(2)}$的同态,其中$\mathrm{SU(2)}$是二维特殊幺正群.
	
	
	\newpage
	
	
	
\subsection*{1 刚体动力学}
	
	这部分其实普物里面都学过了,实在是trivial到不能再trivial,如果再扯一通就是亵渎欧姆尼赛亚了,所以不重要的就跳过了(详见舒力\cite{ShuYouSheng})
	
	有一点新的是刚体的惯量张量$I$是一个实对称三阶张量,必然可以正交对角化(这保证了对线性的角动量和二次型的动能都构成了合法的对角化).也就是说,在刚体上存在一组基矢,使得惯量张量在该基矢下是对角张量.这些方向称为刚体惯量张量的\textbf{主轴方向}.
	
	在主轴方向下,刚体的角动量和转动动能都有非常简洁优美而神圣的形式:
	
	\[L= I_i\Omega_i,\quad T_{\mathrm{rot}}=\frac{1}{2}I_i\Omega_i^2\]
	
	现在我们开始讨论具体的动力学.显然质心的运动是trivial的,我们直接讨论转动的部分.
	
	一般地,我们有
	
	\[\left(\frac{\D L}{\D t}\right)_{\mathrm{space}}=N\]
	
	其中$N$为力矩.
	
	而已知动坐标系的矢量$G$转换到静坐标系对时间的微商有
	
	\[\left(\frac{\D G}{\D t}\right)_{\mathrm{space}}=\left(\frac{\D G}{\D t}\right)_{\mathrm{body}}+\Omega\times G\]
	
	现在取动坐标系为原点位于质心,方向对应惯量主轴方向的坐标系,则有动力学方程
	
	\[
	\begin{aligned}
		&I_{1} \frac{\mathrm{d} \Omega_{1}}{\mathrm{d} t}+(I_{3}-I_{2}) \Omega_{2} \Omega_{3}=N_{1} \\
		&I_{2} \frac{\mathrm{d} \Omega_{2}}{\mathrm{d} t}+(I_{1}-I_{3}) \Omega_{3} \Omega_{1}=N_{2} \\
		&I_{3} \frac{\mathrm{d} \Omega_{3}}{\mathrm{d} t}+(I_{2}-I_{1}) \Omega_{1} \Omega_{2}=N_{3}
	\end{aligned}
	\]
	
	这就是著名的Euler方程,刚体力学的核心.显然这是一个非线性微分方程组,浸润着欧姆尼赛亚不可知的智慧.
	
	然后就是找一大堆在好玩与能解之间达到平衡的例子开始暴算了,什么自由不对称陀螺,对称陀螺的定点运动,进动章动之类的,fx老师一个字都不讲,trivial,trivial(
	
	\newpage
	
	
	
	
\section*{10 数学基础}
	
	试图堆砌数学以掩盖自己摸鱼了大半天的事实.或许会学到简单的流形,然后就跑回正轨.
	
	说实话,数学对物理来说不是必须品,就比如Faraday的力线,不过想到这b东西高中的时候把我恶心了多久,最好还是多学点数学比较好.连话都说不清楚说实话不怎么光荣.
	
\subsection*{0 GTM023\cite{GTM023}节选-仿射空间}
	
	主要是这东西我没看见哪本线代教了,但是Anorld一上来就跟我扯了这个,还是学一下比较好
	
	这里Einstein求和满地乱爬了.
	
	\textbf{点与向量}.设$E$是一个$n$维实线性空间,$A$是一堆(量词)某种元素的结合,$P,Q,...\in A$叫做\textbf{点}.现在我们规定点之间与点和向量之间的关系如下:
	
	1.对于每一个有序对$(P,Q)$存在一个$E$中的向量,称为\textbf{差向量},记为$\overrightarrow{PQ}$.高级一点的话说就是存在一个$A\times A$到$E$的映射.
	
	2.对于任意一个点$P\in A$和$E$中的任意一个向量$x$,存在唯一的$Q\in A$使得$\overrightarrow{PQ}=x$.
	
	3.$\forall P,Q,R\in A$,满足
	
	\[\overrightarrow{PQ}+\overrightarrow{QR}=\overrightarrow{PR}\]
	
	那么$A$就构成了一个$n$维\textbf{仿射空间},$E$称为$A$的\textbf{差空间}.
	
	直观上看,仿射空间可以理解成一个没有坐标轴的线性空间.
	
	第三点的加入给出了$\overrightarrow{PP}+\overrightarrow{PQ}=\overrightarrow{PQ}$,于是$\overrightarrow{PP}=0,\forall P\in A$.从而$\overrightarrow{PQ}=-\overrightarrow{QP}$.同时也易证平行四边形定则$\overrightarrow{P_1P_2}=\overrightarrow{Q_1Q_2}\Leftrightarrow\overrightarrow{P_1Q_1}=\overrightarrow{P_2Q_2}$.
	
	对于任意线性空间$E$,显然动动手就能构造出来一个以之为差空间的仿射空间,直接令仿射空间的元素就是$E$中的向量,差映射就是向量减法即可.
	
	设$A$是一个仿射空间,有一个点$O$被认为是\textbf{原点},那么$A$中的每一点都为向量$x=\overrightarrow{OP}$所确定.显然这样下来由向量$x,y$诱导的两个点之差就是$x-y$.
	
	(从某种角度说,$A$中的每个点都能诱导一个线性空间.好像有流形内味了?)
	
	\textbf{仿射坐标系}.一个仿射坐标系$(O;x_1,\cdots,x_n)$包含了一个固定点$O$,称为\textbf{原点},以及差空间$E$的一组基$x_\nu$,那么$A$中的每一个点$P$都能被$n$元数组$\xi^\nu$表示:
	
	\[\overrightarrow{OP}=\xi^\nu x_\nu\]
	
	其中$\xi^\nu$称为$P$在该仿射坐标系下的\textbf{仿射坐标}.
	
	现在考虑两个仿射坐标系$(O,x_\mu),(O',y_\nu)$,设$x^\mu$到$y^\nu$的过渡矩阵$\alpha_{\nu}^{~\mu}$(原书胡乱使用了求和约定,但是我也不知道这是什么张量),$O'$在前一个仿射坐标系下的坐标$\beta^\mu$,那么有
	
	\[y_\nu=\alpha_{\nu}^{~\mu}x_\mu,\quad\overrightarrow{OO'}=\beta^{\mu}x_{\mu}\]
	
	现在设点$P$在二仿射坐标系下的坐标$\xi^\mu,\eta^\nu$,那么有
	
	\[\overrightarrow{OP}=\xi^{\mu}x_{\mu},\quad\overrightarrow{O'P}=\eta^{\nu}y_{\nu}\]
	
	代入$\overrightarrow{O'P}=\overrightarrow{OP}-\overrightarrow{OO'}$,得
	
	\[\eta^{\nu}y_{\nu}=\xi^{\mu}x_{\mu}-\beta^{\mu}x_{\mu}\]
	
	再代入基变换,得
	
	\[\eta^{\nu}\alpha_{\nu}^{~\mu}x_\mu=\xi^{\mu}x_{\mu}-\beta^{\mu}x_{\mu}\]
	
	于是有
	
	\[\eta^{\nu}\alpha_{\nu}^{~\mu}=\xi^{\mu}-\beta^{\mu}\]
	
	即
	
	\[\eta^{\nu}=\alpha^{\nu}_{~\mu}(\xi^{\mu}-\beta^{\mu})\]
	
	(半吊子指标运算,献丑了orz)
	
	(这不就是逆变矢量马)
	
	\textbf{仿射子空间}.设有一个仿射空间$A$,那么$A$的一个仿射子空间$A_1$是这样的一个$A$的子集:$\forall P,Q\in A_1$,全体向量$\overrightarrow{PQ}$构成的集合是差空间$E$的一个子空间.
	
	设$A_1,A_2$为两个仿射空间,其差空间分别为$E_1,E_2$,那么若$E_1$是$E_2$的子空间或反之,则称$A_1,A_2$\textbf{平行}.易证平行的仿射空间要么一个包含于另一个,要么交集为空.
	
	\textbf{仿射映射}.设$\varphi: P \rightarrow P'$是一个$A$到自身的映射.若该映射满足如下条件:
	
	1.$\overrightarrow{P_{1}Q_{1}} = \overrightarrow{P_{2}Q_{2}}$,则$\overrightarrow{P_{1}'Q_{1}'} = \overrightarrow{P_{2}'Q_{2}'}$,也即$\varphi$保二点之差.
	
	2.由$\varphi$诱导的映射$\psi: \overrightarrow{PQ} \rightarrow \overrightarrow{P'Q'}$是线性映射
	
	那么则称映射$\varphi$是\textbf{仿射映射}.
	
	不难证明给定两个点$O,O'$和线性映射$\psi$,那么存在唯一的仿射映射$\varphi$使得$\varphi(O)=O'$,且$\psi$是其诱导的线性映射.
	
	若在$A$中给定了一个原点$O$,那么所有的仿射映射都具有如下形式:
	
	\[\varphi: x \rightarrow \psi(x) + b\]
	
	若一个仿射变换诱导的线性变换是单位变换,那么称该仿射变换是一个\textbf{平移}.
	
	
	\textbf{欧氏空间}.(oushi不比Euclidean省事多了?)如果一个仿射空间的差空间上定义了一个正定的内积,那么这个仿射空间称为\textbf{欧氏空间}.
	
	剩下的内容请自己找本线代书看.都讲过了.
	
	\newpage
	
	
	
	
	
	
	
	
	
	
	
\subsection*{1 Legendre变换\cite{Anorld}}
	
	Legendre变换把一个线性空间上的函数变成其对偶空间上的函数.
	
	不妨先讨论一元函数的Legendre变换.设$y=f(x)$为一凸函数,且满足$f''(x)>0$.
	
	那么定义函数$f$的Legendre变换式$g(p)$:对每一个$p$值,函数$F(p,x)=px-f(x)$在$x=x(p)$时有最大值,那么定义$g(p):=F(p,x(p))$.
	
	由驻定条件$\partial F/\partial x=0$可得$x(p)$满足$p=f'(x)$,也即$x(p)=(f')^{-1}(p)$.凸函数保证了Legendre变换的存在唯一性.
	
	不难证明:Legendre变换具有对合性.即函数进行两次Legendre变换后得到自己.
	
	Legendre变换的对偶性:若两个函数$f,g$互为Legendre变换,那么它们在Young的意义下对偶:
	
	\[x p \le f(x)+g(p)\]
	
	这由Legendre变换的定义立得.
	
	(不过欧姆尼赛亚在上啊,这跟对偶空间的对偶有半毛钱关系吗?)
	
	至于多元函数的Legendre变换,与之完全相似,就不赘述了.
	
	实话实说,还不如说"把函数'重参数化'到对某些变量的偏导数上"来得简练.......
	
	Legendre变换在理论力学中的一大应用就是从Lagrange力学导出Hamilton力学.
	
	\newpage
	
	
	
	
	
\subsection*{10 群(超级青春版)}
	
	zyf老师的笑话:当年我还在上大学的时候,我们课上群论,结果上课的时候来了个人,从头到尾一直记,非常认真,令数院学生都叹为观止.结果下课的时候该人愤怒地破口大骂:"什么群论,净是数学!"我才明白他可能把这门课理解成什么"群众运动论"了.........

	这里一部分参考了zyf官方推荐的张贤科高等代数学\cite{ZhangXianKe},讲的挺好的,很适合物理壬.....然后还会塞一部分刘川理力\cite{LiuChuan}的附录,可能还会塞李新征老师的群论或者直接抽代......我也不知道要哪几本,物理的觉得太浅,数学的看不懂orz
	
	这里也就图个乐,看看概念了解一下,真要学物院还有两门研本课李群与李代数,群论在物理学中的运用.........orz
	
	\newtheorem*{Group}{定义}
	
	\begin{Group}
		
		一个\textbf{群}(group)是一个非空集合$G$,并且定义了一个$G \times G \to G$的二元映射(*),满足:
		
		(1)结合律:$a * (b * c) = (a * b) * c, \forall a,b,c \in G$
		
		(2)存在幺元(单位元):$\exists e \in G, s.t. e * a = a, \quad \forall a \in G$
		
		(3)存在逆元:$\forall a \in G, \exists b \in G, s.t. \quad b * a = e$.常记$b = a^{-1}$.
		
	\end{Group}
	
	上述群常被记为$(G , *)$.有时也称运算$(*)$为"乘法运算,同理$*$也常常被省略不写.此外注意,上述定义中的幺元与逆元都是左乘意义下的,但可以证明右乘的情况具有相同的结果.
	
	如果一个群还满足交换律,即$ab=ba$,那么称该群为交换群或Abel群.此时群上定义的运算也被称为"加法",幺元记为$0$,逆元称为$-a$.
	
	在应用中,群的定义与"操作"非常贴合.例如三维特殊正交群$\mathrm{SO(3)}$是三维空间中全体旋转变换构成的集合,其中定义的乘法就是两个旋转变换的复合.不难验证这个乘法满足群的定义.
	
	群之间有两个重要的关系:同构与同态.
	
	\newtheorem*{homomorphism}{定义}
	
	\begin{homomorphism}
		
		设有两个群$G,G_1$.若存在一个$G \to G_1$的映射$\varphi$保持群运算,即满足如下条件:
		
		\[\forall a,b \in G, \varphi(ab) = \varphi(a) \varphi(b) \]
		
		那么称$\varphi$是由$G$到$G_1$的一个\textbf{群同态}.
		
		若$\varphi$既单又满,那么称之为\textbf{同构}.
		
	\end{homomorphism}
	
	可以证明,同态的一定将单位元映为单位元,并将逆元映为逆元.由此可见,从某种意义上说,群比群的同态群更小(意会!).
	
	如果群的同态是从群$G$到某个线性空间$V$上的一般线性群$\mathrm{GL(V)}$的映射,那么我们称之为群$G$的\textbf{表示}.换句话说,群的表示是将群的元素与线性变换的矩阵对应起来.
	
	意识到写这些没多大意义,有关的内容本来就已经很通俗易懂了,那就算了吧(
	
	\newpage
	
	
	
	
	
	
	
\subsection*{11 拓扑空间\cite{LiangKunMiao}}
	
	梁老师的这本书实在是物理壬居家旅行杀人放火必备神器,从实数的性质一直延伸到广义相对论和啥啥奇怪的时空无所不包,例子还特多,很适合物理壬看不懂证明只会看例子的习惯(逃)
	
	本来想单刀直入微分流形,结果发现自己已经把开集是啥给忘掉了(悲),简直对不起自己高数考的93了
	
	那就重新开始吧.
	
	首先来点映射:考虑一个映射$f:~X\to Y$,若$Y$中的任意元素的逆像不多于一个(可以没有),则称$f$是\textbf{一一的}(one-to-one);若$Y$中的任意元素都有至少一个逆像,则称$f$是\textbf{到上的}(onto).
	
	在高等数学(分析)中我们知道可以将$\mathbb{R}^n$的子集按照开集与非开集划分.自然地,我们会试图把这个划分方法推广到其他空间中的集合上.然而在其他集合上好在我们可以从$\mathbb{R}^n$中开集的定义中抽象出几个更为"本质"的性质.
	
	\begin{enumerate}
		\item $\mathbb{R}^n$和空集都是开集
		\item 有限个开集的交仍然是开集
		\item 任意个开集的并都是开集
	\end{enumerate}
	
	把这三个性质推广,我们就可以给任意的集合定义开子集的概念.定义了开子集的集合叫\textbf{拓扑空间}.用与机神最为接近的抽象数学语言写出来就是:
	
	\newtheorem*{topology}{定义}
	
	\begin{topology}
		
		非空集合$X$的一个\textbf{拓扑}(topology)$\mathscr{T}$是$X$的若干子集的集合,满足:
		
		\begin{enumerate}
			
			\item $X, \emptyset \in \mathscr{T}$
			
			\item 若$O_i \in \mathscr{T}$,则$\bigcap_{i=1}^n O_{i} \in \mathscr{T}$.
			
			\item 若$\forall \alpha, O_{\alpha} \in \mathscr{T}$,则$\bigcup_{\alpha} O_{\alpha} \in \mathscr{T}$,这里$\alpha$所在的指标集可以是无穷集.
		\end{enumerate}
		
	\end{topology}
	
	回到人话,如果我们能找到一种符合前述的三个条件的从集合$X$中找出特定子集的方法,那么我们可以说找到了$X$的一个拓扑.
	
	\newtheorem*{topologicalSpace}{定义}
	
	\begin{topologicalSpace}
		
		指定了拓扑$\mathscr{T}$的集合$X$称为\textbf{拓扑空间}(topological space).设$O$是拓扑空间$X$的子集,若$O \in \mathscr{T}$,则称$O$为\textbf{开子集}(简称\textbf{开集}).
		
	\end{topologicalSpace}
	
	常见的拓扑有离散拓扑,凝聚拓扑等,但是这几个都是非常trivial的拓扑,不写了(
	
	类似于线性子空间,我们自然想研究$X$子集上的拓扑.设$(X,\mathscr{T})$是拓扑空间,$A$为$X$的任一非空子集,当然我们也可以指定$A$上的拓扑$\mathscr{S}$(这是S),使$A$成为拓扑空间.设$\mathscr{S} = \{V \subset A | V \in \mathscr{T}\}$是自然的.但是如果$A \notin \mathscr{T}$,那么这个定义与拓扑的第一条性质矛盾.一个巧妙的定义是:
	
	\[\mathscr{S} := \{V \subset A | \exists O \in \mathscr{T} s.t. V = A \cap O\}\]
	
	或者等价的形式:
	
	\[\mathscr{S} := \{A \cap V | V \in \mathscr{T}\}\]
	
	可以证明,即使$A\notin\mathscr{T}$,$\mathscr{S}$也构成了$A$的一个拓扑.
	
	此时$\mathscr{S}$称为$A$的由$\mathscr{T}$导出的\textbf{诱导拓扑}.$(A,\mathscr{S})$称为$(X,\mathscr{T})$的\textbf{拓扑子空间}.
	
	现在我们给出一般拓扑空间中连续映射的两个等价定义:
	
	\newtheorem*{continuous}{定义}
	
	\begin{continuous}
		
		a. 设$(X,\mathscr{T}),(Y,\mathscr{S})$为拓扑空间,若$\forall O \in \mathscr{S}, f^{-1}[O] \in \mathscr{T}$,则称$f$是\textbf{连续的}(continuous).
		
		b. 设$(X,\mathscr{T}),(Y,\mathscr{S})$为拓扑空间,若$\forall G' \in \mathscr{S}$且$f(x) \in G'$,有$\exists G \in \mathscr{T}$且$f[G] \subset G'$,则称$f$\textbf{在点$x$处连续}.若$f$在所有的$x \in X$连续,则称$f$\textbf{连续}.
		
	\end{continuous}
	
	前一个定义翻译成人话就是"开集的原象是开集".老数分习题了.后一个仔细想一下$\mathbb{R}$中的情况与欧氏范数,就能发现是一致的.
	
	接下来是一些推广.
	
	\newtheorem*{homeomorphic}{定义}
	
	\begin{homeomorphic}
		设$(X,\mathscr{T}),(Y,\mathscr{S})$为拓扑空间,若存在一一到上的映射$f~:X \to Y$,且$f,f^{-1}$都连续,那么称$(X,\mathscr{T}),(Y,\mathscr{S})$\textbf{互相同胚}.此时映射$f$称为$(X,\mathscr{T})$到$(Y,\mathscr{S})$的\textbf{同胚映射},简称\textbf{同胚}(homeomorphism)
	\end{homeomorphic}
	
	\newtheorem*{neighborhood}{定义}
	
	\begin{neighborhood}
		
		设$x \in X$,$N \subset X$,若$\exists O \in \mathscr{T} s.t. x \in O \subset N$,则称$N$是$x$的一个\textbf{邻域}(neighborhoood).
		
		再设$A \subset X$,若$\exists O \in \mathscr{T} s.t. A \subset O \subset N$,则称$N$是$A$的一个\textbf{邻域}.
		
	\end{neighborhood}
	
	还有有一个定理:
	
	\newtheorem*{neighborhoodEquivalence}{定理}
	
	\begin{neighborhoodEquivalence}
		$A$是开集$\Leftrightarrow$ $\forall x \in A, A$是$x$的邻域
	\end{neighborhoodEquivalence}
	
	可见,以上开集的定义与数分上用$\varepsilon-\delta$的定义是一致的.
	
	\newtheorem*{closedSet}{定义}
	
	\begin{closedSet}
		设$C \subset X$,若$C$的补集$-C \in \mathscr{T}$,则称$C$是\textbf{闭集}(closed set).
	\end{closedSet}
	
	\newtheorem*{closetSetProperty}{定理}
	
	\begin{closetSetProperty}
		
		闭集具有如下性质:
		
		\begin{enumerate}
			
			\item 任意个闭集的交集是闭集
			
			\item 有限个闭集的并集是闭集
			
			\item $X$和$\emptyset$是闭集
			
		\end{enumerate}
		
	\end{closetSetProperty}
	
	\newtheorem*{connected}{定义}
	
	\begin{connected}
		若拓扑空间$(X,\mathscr{T})$除了$X$和$\emptyset$以外没有既开又闭的子集,则称它是\textbf{连通的}(connected).
	\end{connected}
	
	需要指出的是与直观感受更为一致的是道路连通的概念,它与刚刚定义的连通有"微妙的区别"(一般的拓扑空间中道路连通是比连通更强的条件,不过对于物理壬最喜欢的流形来说二者是的等价的)
	
	\newtheorem*{closure}{定义}
	
	\begin{closure}
		集合$A$的\textbf{闭包}(closure)$\overline{A}$是所有含$A$的闭集的交集,即
		\[\overline{A} := \bigcap_{C \supset A, C\text{为闭集}} V\]
	\end{closure}
	
	\newtheorem*{interior}{定义}
	
	\begin{interior}
		集合$A$的\textbf{内部}(interior)$\im(A)$是$A$所有开子集的并集,即
		\[\im(A) := \bigcap_{O \subset A, O \in \mathscr{T}} O\]
	\end{interior}
	
	\newtheorem{boundary}{定义}
	
	\begin{boundary}
		集合$A$的\textbf{边界}(boundary)$\dot{A}:=\overline{A}-\im(A)$,其中的元素称为\textbf{边界点}.$\dot{A}$也记为$\partial A$.
	\end{boundary}
	
	可以证明,$\overline{A}, \im(A), \dot{A}$具有跟数分书上写的$\mathbb{R}^n$中的一样的性质(懒得写了).
	
	\newtheorem*{openCover}{定义}
	
	\begin{openCover}
		设$A \subset X$,若$X$的开子集的集合$\{O_{\alpha}\}$满足$A \subset \bigcap_{\alpha} O_{\alpha}$,那么称$X$是$A$的一个\textbf{开覆盖}(open cover).
	\end{openCover}
	
	这章本来后面还有一节紧致性的选读内容,不过既然是选读那应该就跟物理关系不大,那就...........
	
	
	\newpage	
	
	
	
	
	
	
\subsection*{100 流形和张量场}
	
	挂一波Anorld,\quad 四\quad 句\quad 话\quad 学\quad 会\quad 微\quad 分\quad 流\quad 形
	
	可能这就是神圣的红色(不但是soviet的颜色,还是神圣火星的颜色)数理大贤者吧
	
	所以还是按照梁老师的书来吧(
	
	发现b站上梁灿彬老师的网课讲的比他的书更容易理解,一部分内容也参考了网课上的内容.
	
\subsubsection*{0 微分流形}
	
	好了,大\quad 的\quad 药\quad 来\quad 惹\quad !
	
	流形的背景就是物理壬最喜欢的要多少阶可微就多少阶可微的"连续统",类似于曲线,曲面还是什么超平面之类的东西.显然这些东西都应该有拓扑结构,但这还不够.拓扑空间上只定义了开集,但是平常我们都顺手点来点去(指内积)切来切去(指切触)去的,为了实现这种物理壬最喜欢的东西,我们还得在上面引入微分结构.于是我们就有了微分流形.
	
	\newtheorem*{easyManifold}{疋㐅\cite{Hassani}}
	
	\begin{easyManifold}
		一个微分流形,就是一个点集,集合里面的每一个点都要多光滑有多光滑.并且每一个点的局部看上去都像个$n$维空间.比如一个球面,一个Mobius带等等.
	\end{easyManifold}
	
	虽然这个定义是对机械神的严谨的严重亵渎,但是在被神圣的等离子炮蒸发到数学地狱之前我们还是可以用这个说法对流形有个直观的认识.
	
	\newtheorem*{differentiableManifold}{定义}	
	
	\begin{differentiableManifold}
		
		设有一个拓扑空间$M$,若$M$有开覆盖$\{O_{\alpha}\}$,且满足:
		
		\begin{enumerate}
			
			\item 存在$V_{\alpha} \subset \mathbb{R}^n$且$V_{\alpha}$在通常拓扑下为开集,使得每一个$O_{\alpha}$都与$V_{\alpha}$有同胚$\psi_{\alpha}~:O_{\alpha} \to V_{\alpha}$
			
			\item (\textbf{相容性(compatibility)条件})若开覆盖中的两个开集$O_{\alpha} \cap O_{\beta} \ne \emptyset$,则复合映射$\psi_{\beta} \circ \psi_{\alpha}^{-1}$是$C^{\infty}$(光滑)的.
			
		\end{enumerate}
	
		则称$M$为\textbf{$n$维微分流形}($n$-dimensional diffeerentiable manifold).
		
	\end{differentiableManifold}
	
	(不太明白这里取$\mathbb{R}^n$子集的意义?)
	
	\begin{figure}[H]
		\centering
		\includegraphics[width=0.7\linewidth]{pics/QQ截图20210725215644}
		\caption{梁老师板书(2005年,av画质),示意了各个映射的关系}
		\label{fig:qq20210725215644}
	\end{figure}
	
	这里维数的意义更应该在"自由度"的意义下去理解.比如按照刚才的定义,$\mathbb{R}^3$中的球面应是一个二维流形,而不是三维的.
	
	注意到$\psi_{\beta} \circ \psi_{\alpha}^{-1}$是$V_{\alpha}$到$V_{\beta}$两个$\mathbb{R}^n$中子集的映射,是一个$n$元向量函数,故上述定义是自洽的.
	
	虽然在流形上尚未定义坐标系,但是注意到$O_{\alpha}$与$V_{\alpha} \subset \mathbb{R}^n$同胚,故一个很自然的做法是将流形上点$p \in O_{\alpha}$的\textbf{坐标}定义为$\mathbb{R}^n$中$\psi_{\alpha}(p)$的坐标$x^{\mu}$.这样,我们说$(O_{\alpha},\psi_{\alpha})$构成了一个(局域)\textbf{坐标系}(coordinate system),其\textbf{坐标域}(coordinate patch)为$O_{\alpha}$.如果同时有$p \in O_{\beta}$,那么$p$可以有另一个坐标$x'^{\mu}$.此时,$n$元向量函数$\psi_{\beta} \circ \psi_{\alpha}^{-1}$就叫做\textbf{坐标变换}(coordinate transformation).
	
	这是光滑流形的一般定义,不过物理上的流形性质比这还要好:作为拓扑空间,$M$是Hausdorff的和第二可数的.(看不懂)
	
	\newtheorem*{chart}{定义}
	
	\begin{chart}
		坐标系$(O_{\alpha},\psi_{\alpha})$在数学上又叫\textbf{图}(chart).
	\end{chart}
	
	显然这跟图论里面的图(graph)没有半毛钱关系.
	
	\newtheorem*{altas}{定义}
	
	\begin{altas}
		满足微分流形定义的全体图的集合$\{(O_{\alpha},\psi_{\alpha})\}$称为\textbf{图册}(altas).
	\end{altas}
	
	一个图册中的任意两个图都是相容的.
	
	现在我们可以设想有两个图册$\{(O_{\alpha},\psi_{\alpha})\},\{(O'_{\beta},\psi'_{\beta})\}$能够将拓扑空间$M$定义成流形.这时有两种可能:
	
	1.两个图册互不相容,也就是存在$O_{\alpha}$和$O'_{\beta}$使得$O_{\alpha} \cap O'_{\beta} \ne \emptyset$,且$\psi_{\alpha}$和$\psi'_{\beta}$不满足相容性条件.这时我们说两个图册把$M$定义为两个\textit{不同的}微分流形,并且说这两个图册代表着不同的\textbf{微分结构}.
	
	2.两个图册是相同的.这时我们说它们把$M$定义成同一个流形.于是我们干脆把所有跟这两个图册相容的图册都取并集,构造出一个最大的图册.这样我们就在流形$M$上进行任意坐标变换.
	
	\newtheorem*{CrMap}{定义}
	
	\begin{CrMap}
		设$M,M'$是两个微分流形,若$f~: M \to M'$满足$\psi'_{\beta} \circ f \circ \psi_{\alpha}$是$C^{r}$的,则称$f$为\textbf{$C^{r}$类映射}.
	\end{CrMap}

	这个定义表明流形之间映射的连续程度由复合到对应的$\mathbb{R}^n$子集映射的连续程度决定.
	
	相容性表明上述定义与坐标系的选择无关.
	
	\newtheorem*{diffeomorphicToEachOther}{定义}
	
	\begin{diffeomorphicToEachOther}
		
		设$M,M'$为微分流形,若$\exists f~: M \to M'$,满足:
		
		\begin{enumerate}
			
			\item $f$是一一到上的
			
			\item $f,f^{-1}$是$C^{\infty}$的
			
		\end{enumerate}
		
		那么称$M,M'$\textbf{互为微分同胚}(diffeomorphic to each other),$f$称为$M$到$M'$的\textbf{微分同胚映射},简称\textbf{微分同胚}(diffeomorphism).
		
	\end{diffeomorphicToEachOther}
	
	微分同胚是流形间映射的最高要求.
	
	若$M' = \mathbb{R}$,我们称此时的微分同胚$f$为$M$上的\textbf{函数}(function)或$M$上的\textbf{标量场}(scalar field).若$f$是$C^{\infty}$的,则成为$M$上的光滑函数.$M$上全体光滑函数的集合记为$\mathscr{F}_{M}$或$\mathscr{F}$.今后用到的函数都是这种函数(物理壬狂喜!).
	
	
	
	
	
	\newpage
	
\subsubsection*{1 切矢和切矢场}
	
	我们在普物的学习中已经充分掌握了矢量以及矢量的微积分.现在我们想要把这种有用的工具推广到流形上.
	
	一开始我的想法很简单,微分流形上的开集到$\mathbb{R}^n$的子集有同胚映射,那么直接把$\mathbb{R}^n$的向量按照同胚映射映到流形上不就完了....然后我意识到图之间的坐标变换不一定是线性的,从而在坐标变换下向量的运算性质无法得到保证,所以还是得用其他的方法.这也表明了在一般的流形上"空间"和向量场不能像一般的线性空间中视为等同.具体而言,问题出在同胚上-$n$维流形可能与$\mathbb{R}^n$不同胚.
	
	为了把矢量的概念推广到流形上,我们应该找出矢量的一些更为抽象而普适的性质.物理上,矢量是有大小,有方向的量.在$\mathbb{R^n}$的情况中,每一个矢量都可以按它的方向诱导一个多元函数的方向导数,而大小可以通过将矢量的模视为单位长度来体现(有点绕,但是我觉得是合理的).这样我们就可以进行定义:
	
	\newtheorem*{vectorOnManifold}{定义}
	
	\newtheorem*{vectorSpaceOnManifold}{定理}
	
	\begin{vectorOnManifold}

		设$v$是$\mathscr{F}_M \to \mathbb{R}$的映射,如果$\forall f,g \in \mathscr{F}_M, \alpha,
		\beta \in \mathbb{R}$,$v$满足:
		
		\begin{enumerate}
			
			\item (线性性) \[v(\alpha f + \beta g) = \alpha v(f) + \beta v(g)\]
			
			\item (Leibnitz律) \[v(f g) = f(p) v(g) + g(p) v(f)\]
			
		\end{enumerate}
		
		则称映射$v$为点$p \in M$的一个\textbf{矢量}(vector).
		
	\end{vectorOnManifold}
	
	\begin{vectorSpaceOnManifold}
		
		记$V_{p}$为$n$维流形$M$中$p$点的所有矢量的集合,那么$V_{p}$在如下定义下构成线性空间,且$\dim V_{p} = n$:
		
		\begin{enumerate}
			
			\item \[(v_{1} + v_{2})(f) = v_{1}(f) + v_{2}(f), \forall f \in \mathscr{F}_{M}, v_1,v_2 \in V_{p}\]
			
			\item \[(\alpha v)(f) = \alpha \cdot v(f), \forall f \in \mathscr{F}_{M}, v \in V_{p}, \alpha \in \mathbb{R}\]
			
			\item 定义零元$\underline{0}$满足
			\[\underline{0} (f) = 0, \forall f \in \mathscr{F}_{M}\]
			
		\end{enumerate}
		
	\end{vectorSpaceOnManifold}
	
	\begin{proof}
		
		由上述定义易验证构成线性空间.下证维数为$n$.
		
		任选坐标系使坐标域含$p$.现在考虑如下的$n$个矢量:
		
		\[X_{\mu}(f) := \left.\frac{\partial}{\partial x^{\mu}} (f \circ \psi^{-1})\right|_{p}\]
		
		其中$\psi$为坐标域到$\mathbb{R}^n$子集的同胚.
		
		现在我们来证明它们线性无关.设
		
		\[\alpha^{\mu}X_{\mu} = 0\]
		
		现考虑函数$f_{\nu} \circ \psi^{-1}~:(x^1,x^2,\cdots,x^n) \to x^{\nu}$,将上一等式作用于$f_{\nu}$得
		
		\[RHS = 0\]
		
		\[LHS = \alpha^{\mu} \left.\frac{\partial x^{\nu}}{\partial x^{\mu}}\right|_{p} = \alpha^{\mu} \delta^{\nu}_{~\mu} = \alpha^{\nu}\]
		
		故$\alpha^{\nu} = 0$,即$X_{\mu}$线性无关.
		
		而最难的部分是证明全空间的元素都可以由$X_{\mu}$线性表出.这里我们直接
		
		$\mathrm{import~Wald.General\_Relativity.P16}\cite{Wald}$
		
		得出结论
		
		\[\forall v \in V_{p}, v = v^{\mu} X_{\mu}, \text{where}~v^{\mu} = v(f_{\mu}).\]
		
		故$\dim V_{p} = n$.
		
	\end{proof}
	
	关于不同坐标系间的坐标变换,我们有如下定理:
	
	\newtheorem*{coordinateTransform}{定理}
	
	\begin{coordinateTransform}
		
		设$\{x^\mu\},\{x'^{\nu}\}$为两个坐标系且坐标域交集非空,$p$为交集中的一点,则坐标分量满足变换关系
		
		\[v'^{\nu} = \left.\frac{\partial x'^{\nu}}{\partial x^{\mu}}\right|_{p} v^{\mu}\]
		
	\end{coordinateTransform}
	
	这由锁链法则与简单的线性代数易得.
	
	现在我们讨论曲线与曲线的切矢.自然地可以定义\textbf{曲线}(curve):$l~:I \to M$,其中$I$为$\mathbb{R}$上的区间.无特殊说明下$l$都指光滑曲线,即$C^{\infty}$的.需要注意的是,这里的曲线与直观上的曲线不同于一般我们"看到"的曲线是流形的子集,但这里定义的曲线是映射.在保像集的前提下将$I$非退化地映为$I'$,叫做曲线的\textbf{重参数化}(这相当于给直观的曲线换一套参数表示).
	
	坐标系中,$\psi^{-1}$在$x^{\nu}$不变$(\nu \ne \mu)$下的限制映射称为$x^{\mu}$坐标线.
	
	\newtheorem{tangentVector}{定义}
	
	\begin{tangentVector}
		
		设$C(t)$是流形$M$上的曲线,则线上$C(t_{0})$点的切于曲线$C$的\textbf{切矢}(tangent vector)是$C(t_{0})$点的矢量,满足:
		
		\[T(f) = \left.\frac{\D(f \circ C)}{\D t}\right|_{t_0}\]
		
		切矢也常记为$\left.\frac{\partial}{\partial t}\right|_{C(t_0)}$.
		
	\end{tangentVector}
	
	注意到$x^{\mu}$坐标线的基矢作用在任意的$f$上正好就是$\left.\frac{\partial (f \circ \psi^{-1})}{\partial t}\right|_{C(t_0)}$,从中可以看出上述记法的合理性.
	
	在流形上我们称线性相关的二矢量为\textbf{互相平行的}(parallel).
	
	显然$V_{p}$中的任意元素都可以视为过$p$的某曲线的切矢.因此$p$点的矢量也称为\textbf{切矢量}(tangent vector),$V_{p}$称为$p$点的\textbf{切空间}(tangent space).
	
	由定义可以看出,切空间与"位形"空间是对偶构造的.
	
	\newtheorem*{vectorField}{定义}
	
	\begin{vectorField}
		设$A \subset M$,若给$A$中任意一点$p$指定一个$V_{p}$中的矢量,就得到一个定义在$A$上的\textbf{矢量场}(vector field).
	\end{vectorField}
	
	不妨设$f$是$M$上的函数,$v$是$M$上的矢量场,则$v$在任意一点$p$的值$v|_{p}$将$f$映到一个实数$v|_{p}(f)$.因此,矢量场可以视为把$f$映成函数$v(f)$的映射.
	
	\newtheorem{CrVectorField}{定义}
	
	\begin{CrVectorField}
		若$M$上的矢量场$v$作用到任意$C^{\infty}$类函数得到$C^{r}$类函数,则称$v$是\textbf{$C^{r}$类的}.若得到光滑函数,则称$v$是\textbf{光滑的}.
	\end{CrVectorField}
	
	坐标基矢$X_{\mu} = \frac{\partial}{\partial x^{\mu}}$构成坐标域上的$n$个光滑矢量场,称为\textbf{坐标基矢场}.
	
	\newtheorem{integralCurve}{定义}
	
	\begin{integralCurve}
		曲线$C(t)$叫矢量场$v$的\textbf{积分曲线}(integral curve),若其上每一点的切矢都等于该点的$v$值.
	\end{integralCurve}
	
	\newtheorem*{uniquenessOfIntegralCurve}{定理}
	
	\begin{uniquenessOfIntegralCurve}
		设$v$是$M$上的光滑矢量场,则$\forall p \in M$,必有$v$的唯一的积分曲线$C(t)$经过$p$.
	\end{uniquenessOfIntegralCurve}
	
	除了这里已经写的,原书还涉及了一些群论的内容,但看上去跟物理没什么关系.接下来是对偶矢量与张量场,这些在线性代数课中都已经学过了,便不再赘述.
	
	(准确来说是数学读不下去了,学了流形以为自己可nb了跑回Anorld结果发现Lagrange力学的构造还没学完呢,更不用说Hamilton了(悲))
	
	ToDo:本章剩下的部分补充一下,Lie群,Lie代数,群,群表示,微分形式,流形上的Stokes公式,辛几何(欧姆尼赛亚在上啊,这要学到什么时候?)
	
	
	
	
	
	
	\newpage
	
	
\section*{11 Hamilton力学}
	
	如果只是限制在经典力学的框架里面,那么Lagrange力学和Newtond的矢量力学并没有什么高下之分,顶多是Lagrange力学对于有约束问题更方便一些.非矢量形式的力学,最主要的优势在于对更广阔的的物理体系的适用性.所以我们再介绍一种新的力学表述:Hamilton力学.
	
\subsection*{0 Hamilton正则方程的经典构造}
	
	我说怎么看了半天感觉这么膈应,结果突然意识到这tm又退回纯经典了......
	
	经典力学中,Lagrange函数$L$是一个$\mathbb{R}^n \times \mathbb{R}^n \times \mathbb{R}$到$\mathbb{R}$的函数,前$n$个自变量为$q_{i}$,接着是$\dot{q}_{i}$,最后是$t$.其全微分
	
	\[\D L = \frac{\partial L}{\partial q_{i}} \D q_{i} + \frac{\partial L}{\partial \dot{q}_{i}} \D \dot{q}_{i} + \frac{\partial L}{\partial t} \D t\]
	
	考虑到一般情况下动能$T$都是$\dot{q}_{i}$的正定二次型, 我们可以将$L$对$\dot{q}_{i}$进行 Legendre 变换:
	
	记对偶变元
	
	\[p_{i} := \frac{\partial L}{\partial \dot{q}_{i}}\]
	
	以及对偶函数
	
	\[H = p_{i}\dot{q}_{i} - L\]
	
	则有全微分
	
	\[\begin{aligned}
		\D H &= p_{i} \D \dot{q}_{i} + \dot{q}_{i} \D p_{i} - \D L\\
			 &= \dot{q}_{i} \D p_{i} - \frac{\partial L}{\partial q_{i}} \D q_{i} - \frac{\partial L}{\partial t} \D t\\
			 &= \dot{q}_{i} \D p_{i} - \dot{p}_{i} \D q_{i} - \frac{\partial L}{\partial t} \D t
	\end{aligned}\]
	
	最后一步利用了无广义力(系统可被 Lagrange 函数完全描述)情况下的 Lagrange 方程$\frac{\partial L}{\partial q_{i}} - \frac{\D}{\D t}\frac{\partial L}{\partial \dot{q}_{i}}$.
	
	考虑到Lagrange方程已在最后一步中得到体现,因此我们可以说如下从全微分关系导出的方程组能够等价地描述力学系统的演化
	
	\[\begin{cases}
		\dot{q_{i}} = \frac{\partial H}{\partial p_{i}}\\
		\dot{p_{i}} = -\frac{\partial H}{\partial q_{i}}
	\end{cases}\]
	
	这个方程组称为\textbf{Hamilton正则方程}( canonical equations of Hamilton ).对偶函数$H~: \mathbb{R}^n \times \mathbb{R}^n \times \mathbb{R} \to \mathbb{R}$称为\textbf{Hamilton函数}( Hamiltonian ).显然, Hamilton 正则方程作为一组形式十分对称的一阶微分方程组,比 Lagrange 方程优美而神圣多了,这就是为什么称为"正则"方程.
	
	结合正则方程,从$H$的全微分里面我们还可以得到
	
	\[\frac{\D H}{\D t} = \frac{\partial H}{\partial p_{i}} \dot{p}_{i} - \frac{\partial H}{\partial q_{i}} \dot{q}_{i} - \frac{\partial L}{\partial t} \D t = \frac{\partial H}{\partial t}= - \frac{\partial L}{\partial t}\]
	
	可见,Lagrange函数不含时间时$H$为常数,对应能量守恒.
	
	从正则方程的对称形式中可以看出,在 Hamilton 力学的框架中, "坐标" 和 "动量" 的概念不是一成不变的,而是可以"互换"的(刘川书上的,意会是意会到了,但具体还是不太懂).更准确的称呼是广义坐标与广义动量互为\textbf{共轭变量}.一个$n$个自由度的系统,我们有$n$对共轭变量,且系统的运动状态由这$2n$个变量完全描述.以这$2n$个变量为坐标,构成了$2n$为的相空间(从现代的角度上看,考虑到相空间的拓扑结构,更应该称为相流形( phase manifold )).
	
	(写到这里编译一下突然报出一大堆提示,排查了一下发现好像中英混合输入的时候得多打几个空格,不然TeX无法给文本正确分词以进行换行?)
	
	Landau还另外讲了一个 Routh 函数,大意就是如果$q_{j}$为循环坐标,则可以通过单独对$\dot{q}_{j}$进行 Legendre 变换得到新函数$R(p_{j},q_{i},\dot{q}_{i})
	$,这个函数对$p_{i},q_{i}$是 Hamilton 函数,对其他变量是 Lagrange 函数.广义动量守恒可得$\dot{q}_{j}=C$,代入$R$则直接消去$p_{j},q_{j}$这一自由度.从而使系统得到简化.
	
	
	
	\newpage

\subsection*{1 $\xi$符号与Liouville定理}

	Hamilton力学里面广义坐标和广义动量十分平等,所以我们可以把它写成更加对称的形式:
	
	定义变量集$\xi^{j}$:
	
	\[\xi^{j} := 
	\begin{cases}
		q^{j},~j = 1,\cdots,n\\
		p_{j-n},~j = n + 1,\cdots,2n
	\end{cases}\]

	那么我们可以将Hamilton正则方程写成:
	
	\[\dot{\xi}^{j} = \omega^{jk} \partial_{k}H\]
	
	其中$\partial_{k} = \partial / \partial_{k}$,$\omega$为$2n$阶的标准辛矩阵
	
	\[\omega = \begin{pmatrix}0&I_{n}\\-I_{n}&0\end{pmatrix}\]
	
	现在我们来考察相空间体积元$\wedge_{i} \D \xi^{i}$的演化.高数中我们知道$t$时刻的体积元与起始时刻的差一个Jacobi行列式
	
	\[J = \frac{\partial (\xi^{1}(t),\xi^{2}(t),\cdots,\xi^{n}(t))}{\partial (\xi^{1}(0),\xi^{2}(0),\cdots,\xi^{n}(0))}\]
	
	现在我们考虑对应的导数矩阵$M_{ij} = \frac{\partial \xi^{i}(t)}{\partial \xi^{j}(0)}$,显然$J = \det M$.
	
	运用矩阵函数公式$\ln \det M = \mathrm{tr} \ln M$,可得(此公式利用矩阵函数的计算方法易得)
	
	\[\frac{\D J}{\D t} = \frac{\D}{\D t} \E^{\ln \det M} = \frac{\D}{\D t} \E^{\mathrm{tr} \ln M} = J(t) \mathrm{tr} (M^{-1}\dot{M})\]
	
	而
	
	\[\mathrm{tr} (M^{-1}\dot{M}) = \frac{\partial \xi^{i}(0)}{\partial \xi^{j}(t)}\frac{\partial \dot{\xi}^{j}(t)}{\partial \xi^{i}(0)} = \frac{\partial \dot{\xi}^{j}(t)}{\partial \xi^{j}(t)}\]

	代入Hamilton正则方程,得

	\[\frac{\D J}{\D t} = J \frac{\partial \dot{\xi}^{j}(t)}{\partial \xi^{j}(t)} = \omega^{jk} \partial_{j} \partial_{k} H\]

	考虑到$\omega$的反对称性,有
	
	\[\frac{\D J}{\D t} = 0 \]

	也即相空间的体积元守恒.这就是作为统计力学基础的Liouville定理.
	
	特别是对于含时的Hamiltonian,Liouville定理依然成立.事实上,上述推导根本没有涉及Hamiltonian对时间的偏导.
	
	这里是参照刘川老师的证法,Landau的推导用了正则变换,Goldestein\cite{Goldestin}里面用了Possion括号.....或许我后面应该再看看
	
	
	
	
	\newpage
	
	
\subsection*{10 Possion括号}
	
	随便考察一个广义坐标,广义动量与时间的函数(力学量)$f(p,q,t)$,有
	
	\[\frac{\D f}{\D t} = \frac{\partial f}{\partial p_{i}} \dot{p}_{i} + \frac{\partial f}{\partial q_{i}} \dot{q}_{i} + \frac{\partial f}{\partial t}\]
	
	将正则方程代入,得
	
	\[\frac{\D f}{\D t} = -\frac{\partial f}{\partial p_{i}} \frac{\partial H}{\partial q_{i}} + \frac{\partial f}{\partial q_{i}} \frac{\partial H}{\partial p_{i}} + \frac{\partial f}{\partial t}\]
	
	这里出现了一个对称而圣洁的表达式$\frac{\partial f}{\partial q_{i}} \frac{\partial H}{\partial p_{i}} - \frac{\partial f}{\partial p_{i}} \frac{\partial H}{\partial q_{i}}$.仿造它的结构,我们定义力学量$f,g$的\textbf{Possion括号}:
	
	\[[f,g] := \frac{\partial f}{\partial q_{i}} \frac{\partial g}{\partial p_{i}} - \frac{\partial f}{\partial p_{i}} \frac{\partial g}{\partial q_{i}}\]
	
	优美吗?优美的触手佬都要漏机油了好吧(误)
	
	由此可见,力学量$f$是守恒量(运动积分)的充要条件是
	
	\[\frac{\partial f}{\partial t} + [f,H] = 0\]
	
	特别地,如果该力学量本身不显含t,则
	
	\[[f,H] = 0\]
	
	Possion括号的定义也可以用之前定义的$\omega$写成
	
	\[[f,g] := (\partial_{j} g) \omega^{jk} (\partial_{k} f)\]
	
	由此不难看出Possion括号是一个反对称的双线性函数.(或许这跟所谓的辛结构有所关联?然而我看不懂Anorld也不知道orz)
	
	Possion括号还具有如下两个形式:
	
	\begin{enumerate}
		
		\item (Lebnitz性) \[[f_{1}f_{2},g] = f_{1}[f_{2},g]+f_{2}[f_{1},g]\]
		
		\item (Jacobi等式) \[[f,[g,h]] + [g,[h,f]] + [h,[f,g]] = 0\]
		
	\end{enumerate}
	
	利用这些关系,任意两个已知函数的Posssion括号都可以化简为基本的Possion括号:
	
	\[[q_{i},q_{j}] = [p_{i},p_{j}] = 0,\quad [q_{i},p_{j}] = \delta_{ij}\]
	
	Possion括号的另一个重要的性质是\textbf{Possion定理}:如果$f,g$构成了力学系统的两个运动积分,那么它们的Possion括号也是运动积分.
	
	证明是trivial的.只要是额叶没被切掉做成机仆的人应该都能套套公式做出来.
	
	(今天到这了,发现自己菜的安详orz结果现在又有点想跑回去学数学了)

	
	
	
	
	
	
	



	
	
	
	
	
	
	\newpage
	
	\begin{thebibliography}{?}
		\bibitem{LiuChuan} 刘川,理论力学[M],北京,北京大学出版社,2019.09
		\bibitem{Landau} Landau, Lifshitz著,李俊峰,鞠国兴译,力学(第五版)[M],北京,高等教育出版社,2007.04
		\bibitem{LiangKunMiao} 梁昆淼,力学(理论力学部分)[M],不知道哪里,不知道什么出版社,不知道什么时候
		\bibitem{ShuYouSheng} 舒幼生,力学[M],北京,北京大学出版社,2005.09
		\bibitem{GTM023} W.H.Greub, Linear Algebra(Third Edition)[M], New York, Springer, 1967
		\bibitem{Anorld} Anorld著,齐民友译,经典力学的数学方法[M],北京,高等教育出版社, 2006.01
		\bibitem{LiangCanBin} 梁灿彬,周彬著,微分几何入门与广义相对论(上)[M],北京,科学出版社, 2006.01
		\bibitem{ZhangXianKe} 张贤科,许甫华著,高等代数学[M],北京,清华大学出版社, 2004.07
		\bibitem{Hassani} S.Hassani, Mathematical Physics[M], New York, Springer, 2013
		\bibitem{Wald} R.M.Wald, General Relativity[M], Chicago, The University of Chicago Press, 1984
		\bibitem{Goldestin} H.Goldstein et al, Classical Mechanics[M], San Francisco, Addison Wesley, 不知道什么时候
	\end{thebibliography}
	
	
	
	
	
	
\end{document}